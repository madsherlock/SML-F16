\chapter{Introduction}
This report documents the results of the first exercise within
a project concerned with digit recognition.
The purpose of the project is to develop a system capable of recognizing
(classifying) hand written digits (0 - 9) using differend machine learning techniques,
and documenting the performance of the system.
The first exercise, documented here, was to measure the impact
of image density (DPI), smoothing kernel size and parameter \(k\)
of the k-NN algorithm on the misclassification error rate.
The datasets used for training and testing of the system
were hand written by two students (the authors), and each
consist of 400 examples of each digit (4000 digits in each dataset).
See fig. \ref{fig:data} for example data.

\begin{figure}[H]
\centering
\includegraphics[width =0.4\textwidth]{../../SML-database/2016/group2/member1/Ciphers100-0.png}
\caption{Example of the  dataset}
\label{fig:data}
\end{figure} 

The process of digit recognition can be divided into 3 steps:
\begin{itemize}
\item Preprocessing - Extracting the data, and discarding irrelevant information. 
\item Feature extraction - Extracting relevant features.
\item Classification - Using the extracted features for digit classification.
\end{itemize}