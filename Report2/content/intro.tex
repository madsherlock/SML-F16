\section{Introduction}
\label{sec:introduction}
This report documents the results of the second exercise within
a project concerned with digit recognition.
The purpose of the project is to develop a system capable of recognizing
(classifying) hand written digits (0 - 9) using different machine learning techniques,
and documenting the performance of the system.
The purpose of the second exercise, documented here, was to
compare k-Nearest Neighbour classification accuracy when using
data from the same individuals in both training and testing sets,
to when the testing set is from individuals not represented in the training set.
This provides insight into the difficulty of the problem and the important
concept of validation.

A Principal Component Analysis was carried out with the purpose
of reducing the number of dimensions in the data.
The exercise was to select the number of principal components
based on classification performance.

Lastly, the effects of applying data normalization before and after
Principal Component Analysis were to be investigated in this exercise.

The datasets used for training and testing of the system
were hand written by students (the authors included), and each
consist of 400 examples of each digit (4000 digits in each dataset).

In the analyses of data for parameter optimization,
it was chosen to use data from one individual,
so that parameter impact on classification accuracy
could be most accurately measured. This way of parameter
optimization also speeds up the development of the digit recognition system.
When comparing methods, data from several individuals should be used.

Data processing was performed in R \citep{R}, using the caret package \citep{caret},
while MATLAB Student \citep{matlabstudent} was used for visualization.

%\begin{enumerate}
%\item first one
%\item second one
%\end{enumerate}
%
%\begin{figure}[ht]
%\centering
%  \begin{subfigure}[t]{0.3\textwidth}
%    \includegraphics[width = \textwidth]{graphics/sq/dist}
%    \caption{Brushfire based coverage.}
%    \label{sqdist}
%  \end{subfigure}
%  \begin{subfigure}[t]{0.3\textwidth}
%    \includegraphics[width = \textwidth]{graphics/sq/top}
%    \caption{Points that were not covered by wall distance.}
%    \label{sqtop}
%  \end{subfigure}  
%\caption{Offline planning.}
%\label{offline_steps}
%\end{figure}
