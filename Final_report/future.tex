\section{Future work}
Future work should compare classification algorithm implementation
running times more thoroughly.
While the work presented here includes a classification time
complexity analysis of the naïve k-NN algorithm,
a time complexity analysis of the same sort is not presented for the SVM classifier.

Including more data, such as that of the 2015 class,
might have shifted classifier comparison results,
possibly leading to stronger conclusions on
the accuracy of the apparently superior SVM classifier
vs the k-NN classifier.

Future work should also vary the additional parameters of the SVM
classifier, and experiment with sigmoidal, gaussian and higher-degree polynomial
kernels, as their non-linearity might yield better results
than the accuracies obtained here.

Since a standard digit recognition dataset, MNIST \citep{mnist}, exists,
future work should benchmark against that.
This would greatly improve the usefulness
and reusability of the implementations and results
of works such as this.

Of course, it would also be relevant to test other
classification algorithms, such as Convolutional Neural Networks,
Random Forests and Least-Squares SVM.

Future work should consider printing the paper
sheets with a red grid instead of a black one,
and permit students from using a red pen.
This way, color segmentation may be applied
to remove the grid programmatically.

Detecting the grid corners could also be done
automatically, leading to a faster data collection process
less prone to human error.
