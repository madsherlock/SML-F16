\section{Introduction}
Recognition of handwritten digits is a potentially computationally expensive
task, which is solvable by machine learning. The task is
relevant due to its many applications, such as
postal service automation
and
spreadsheet scanning.
The task is well suited for trying out supervised machine learning algorithms
with little setup required for preprocessing and labeling,
and a fairly low-dimensional data representation.

The main purpose of this project is to compare two classification algorithms,
k-Nearest Neighbours (k-NN) and Support Vector Machines (SVM),
for handwritten digit recognition.
The second purpose of this project is to achieve good classification performance,
and thus, some preprocessing is needed.
Lastly, as the digits used in this project
were written by a class of 23 always competitive students,
it was chosen to estimate the quality of the hand written digits
and identify the student with the worst hand writing.
This required a definition of hand written digit quality,
which is also presented here.

The work presented here relies on previous knowledge
and experiences gained from course exercises.