\section{PCA}
Principal component analysis (PCA) is a procedure that reduces the number of 
variables a dataset may contain.  It often useful if it is known beforehand that 
some of the variables are redundant, meaning that some of the variables may be 
correlated with others and not provide new knowledge which would have been 
achieved through some other variables.  PCA reduces the dataset into a smaller 
number of principal component (PC), which will account for most of the variable 
that is observed by the dataset. A PC can be defined as a linear combination of 
weighted observed variables . It it up to the user to determine how many 
components are needed, and which may be redundant. Often is a screeplot used to 
determine the number of component should be used. A screeplot depict the 
variance gained from using increasing numbers of PC, from which is can be 
deduced when adding more PC would be beneficial for the task at hand. \\

\todo[inline]{Maybe remove this part.. it is the stupido...}
The principal component is computed as such: 
First is the mean vector of the whole dataset computed. 
The mean vector is used to center the data, such that the mean of the dataset becomes equal to zero. This is achieved by subtracting the mean from each data vectorh 




