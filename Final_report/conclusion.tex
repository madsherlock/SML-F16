\section{Conclusion}
\label{sec:conclusion}
%The main purpose of this project is to compare two classification algorithms,
%k-Nearest Neighbours (k-NN) and Support Vector Machines (SVM),
%for handwritten digit recognition.
%The second purpose of this project is to achieve good classification performance,
%and thus, some preprocessing is needed.
%Lastly, as the digits used in this project
%were written by a class of 23 always competitive students,
%it was chosen to estimate the quality of the hand written digits
%and identify the student with the worst hand writing.
%This required a definition of hand written digit quality,
%which is also presented here.
k-NN with number of neighbours k=5 achieves significantly higher overall 
classification accuracy than SVM with polynomial kernel of degree 2
with scale factor 0.1 and cost of constraints violation C=0.5,
with \(\alpha=0.05\),
when each algorithm is trained with a dataset of hand written digits
from a class of 23 people, and tested on a dataset
with digits written by the same people.
When the training set only contains data from 22 people,
and the test set is written by a person not in the training
set, k-NN does not perform significantly worse
than SVM, although a lower overall accuracy was observed.
Neither does the k-NN classifier perform significantly worse than SVM
when applied to one persons data at a time,
although a lower overall accuracy was observed.

The highest obtained overall classification accuracy is 0.854 with a standard
deviation of 0.0205, for the SVM classifier.

A successful preprocessing scheme involving the perspective transform,
gaussian blur and Principal Component Analysis was implemented.

Person \(G2M1\) (one of the authors) has the worst digit hand writing of the 2016 SML class,
when the writing quality is measured by classification
error on own data. This means \(G2M1\) has the least
within-digit consistent and between-digit varied
hand writing.